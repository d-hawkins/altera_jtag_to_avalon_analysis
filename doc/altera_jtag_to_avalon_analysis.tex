
\documentclass[10pt,twoside]{article}

% Math symbols
\usepackage{amsmath}
\usepackage{amssymb}

% Headers/Footers
\usepackage{fancyhdr}

% Colors
\usepackage[usenames,dvipsnames]{color}

% Importing and manipulating graphics
\usepackage{graphicx}
\usepackage{subfig}
\usepackage{lscape}

% Misc packages
\usepackage{verbatim}
\usepackage{dcolumn}
\usepackage{ifpdf}
\usepackage{enumerate}

% PDF Bookmarks and hyperref stuff
\usepackage[
  bookmarks=true,
  bookmarksnumbered=true,
  colorlinks=true,
  filecolor=blue,
  linkcolor=blue,
  urlcolor=blue,
  hyperfootnotes=true
  citecolor=blue
]{hyperref}

% Improved citation handling
% (include after the hyperref stuff)
\usepackage{cite}

% Pretty-print code
\usepackage{listings}

% -----------------------------------------------------------------
% Setup the margins
% -----------------------------------------------------------------
% Footer Template

% Set left margin - The default is 1 inch, so the following
% command sets a 1.25-inch left margin.
\setlength{\oddsidemargin}{0.25in}
\setlength{\evensidemargin}{0.25in}

% Set width of the text - What is left will be the right
% margin. In this case, right margin is
% 8.5in - 1.25in - 6in = 1.25in.
\setlength{\textwidth}{6in}

% Set top margin - The default is 1 inch, so the following
% command sets a 0.75-inch top margin.
%\setlength{\topmargin}{-0.25in}

% Set height of the header
\setlength{\headheight}{0.3in}

% Set vertical distance between the header and the text
\setlength{\headsep}{0.2in}

% Set height of the text
\setlength{\textheight}{8.5in}

% Set vertical distance between the text and the
% bottom of footer
\setlength{\footskip}{0.4in}

% -----------------------------------------------------------------
% Allow floats to take up more space on a page.
% -----------------------------------------------------------------

% see page 142 of the Companion for this stuff and the
% documentation for the fancyhdr package
\renewcommand{\textfraction}{0.05}
\renewcommand{\topfraction}{0.95}
\renewcommand{\bottomfraction}{0.95}
% dont make this too small
\renewcommand{\floatpagefraction}{0.35}
\setcounter{totalnumber}{5}

%-----------------------------------------------------------
% Abbreviated symbols
%-----------------------------------------------------------
\newcommand{\sinc}{\ensuremath{\,\text{sinc}}}
\newcommand{\rect}{\ensuremath{\,\text{rect}}}
\newcommand{\floor}{\ensuremath{\,\text{floor}}}

% =================================================================
% The document starts here
% =================================================================
%
\begin{document}
\title{Altera JTAG-to-Avalon Analysis}
\author{D. W. Hawkins (dwh@caltech.edu)}
%\date{\today}
% Date from v1.0 document
\date{February 8, 2012}
\maketitle

% No header/footer on the first page
\thispagestyle{empty}

\tableofcontents

% start the intro on an odd page
\cleardoublepage
%\clearpage

% Set up the header/footer
\pagestyle{fancy}
\lhead{Altera JTAG-to-Avalon Analysis}
\chead{}
\rhead{\today}
\lfoot{}
\cfoot{}
\rfoot{\thepage}
\renewcommand{\headrulewidth}{0.4pt}
\renewcommand{\footrulewidth}{0.4pt}

% Set the listings package language to Tcl
\lstset{language=Tcl}

% =================================================================
\section{Introduction}
% =================================================================

% -----------------------------------------------------------------
% Altera JTAG Interface
% -----------------------------------------------------------------
%
\begin{figure}
  \begin{center}
    \includegraphics[width=0.85\textwidth]{figures/jtag_diagram.pdf}
  \end{center}
  \caption{Altera JTAG (IEEE 1149.1) Interface. JTAG transactions
  consist of writing to the instruction register followed by
  writing to or reading from the data register.}
  \label{fig:jtag_diagram}
\end{figure}
% -----------------------------------------------------------------

Efficient development and debugging of Altera
\href{http://en.wikipedia.org/wiki/Fpga}{FPGA} based
designs requires a detailed knowledge of the
\href{http://en.wikipedia.org/wiki/Jtag}{JTAG} interface.
The Altera FPGA JTAG interface has several uses;
%
\begin{itemize}
\item The Quartus II programmer uses JTAG for FPGA configuration.
\item The SignalTap II logic analyzer interface uses JTAG to communicate
with the SignalTap II logic embedded in the FPGA.
\item The NIOS II debug and JTAG-UART communicate with the NIOS II IDE
via JTAG.
\item The Qsys and SOPC Builder JTAG-to-Avalon-MM bridge component uses
JTAG to generate Avalon-MM master transactions.
\item The SLD Virtual JTAG component~\cite{Altera_Virtual_JTAG_2008}
can be used to create custom components accessible via JTAG.
\end{itemize}
%
The JTAG interface is a {\em synchronous} serial interface consisting of
the JTAG clock (TCK), mode select (TMS), serial data in (TDI), and
serial data out (TDO).
Figure~\ref{fig:jtag_diagram} shows a diagram of the Altera JTAG
interface~\cite{Altera_JTAG_Configuration}, while 
Figure~\ref{fig:jtag_tap} shows the JTAG TAP controller
finite state machine (FSM). 

This document is the {\em missing manual} for Altera's JTAG-to-Avalon
interfaces, it provides simulation details and hardware measurements.

% -----------------------------------------------------------------
% JTAG TAP Controller
% -----------------------------------------------------------------
%
\begin{figure}
  \begin{center}
    \includegraphics[width=0.9\textwidth]{figures/jtag_tap.pdf}
  \end{center}
  \caption{JTAG (IEEE 1149.1) Test Access Port (TAP) Controller.
  State transitions are controlled by the JTAG clock (TCK) and
  mode select (TMS) control.}
  \label{fig:jtag_tap}
\end{figure}
% -----------------------------------------------------------------

\clearpage
% =================================================================
\section{JTAG-to-Avalon Protocols}
% =================================================================
\label{sec:protocol}

The Altera Avalon Specification~\cite{Altera_Interface_Specification_SOPC_2011,Altera_Interface_Specification_Qsys_2011} defines two interfaces;
%
\begin{itemize}
\item Avalon Streaming (Avalon-ST) Interface
\item Avalon Memory-Mapped (Avalon-MM) Interface
\end{itemize}
%
The JTAG-to-Avalon-MM master consists of a JTAG-to-Avalon-ST 
interface that converts JTAG serial transactions into byte streams in and
out of the design, and bytes-to-packets conversion logic that 
decodes and encodes a binary protocol transported over the byte
streams. The binary protocol encodes whether to perform an Avalon-MM
read or write transaction, and the response for each transaction 
type.

Access from the JTAG interface to the FPGA fabric is coordinated
by the {\em System-Level Debug} (SLD) JTAG hub. This hub is automatically
generated during synthesis (the component appears the Quartus II hierarchy 
display). The JTAG-to-Avalon components are based on the Altera
{\em SLD Virtual JTAG} Component~\cite{Altera_Virtual_JTAG_2008}
(they are actually based on a variation of this component that allows
a manufacturer ID, component ID and version to be specified).
The Virtual JTAG component uses two JTAG instructions to create
a Virtual Instruction Register (VIR) and a Virtual Data Register (VDR)
that exist in the FPGA fabric. The JTAG-to-Avalon components use the
VIR for selecting the component operating mode, and use the
VDR path for mode-specific data registers.

% -----------------------------------------------------------------
\subsection{JTAG-to-Avalon-ST}
% -----------------------------------------------------------------
\label{sec:jtag_to_avalon_st}

The JTAG-to-Avalon-ST bridge uses the Virtual JTAG interface to
stream bytes into and out of the FPGA fabric, and to access
control and status registers. 
Table~\ref{tab:jtag_to_avalon_st_vir_codes} shows the VIR modes
and the register bits.

In {\em DATA} mode, the JTAG-to-Avalon-ST bridge generates
byte-streams from the host-to-device and from the device-to-host.
Each byte-stream starts with a 16-bit header, and is followed
by encoded data.
Table~\ref{tab:jtag_to_avalon_st_headers} shows the headers
(see Section~\ref{sec:hardware} for encoding details),
while Table~\ref{tab:jtag_to_avalon_st_protocol_codes}
shows the data protocol. 
The {\em IDLE} code indicates {\em no data} and can appear in 
the host-to-device or device-to-host byte-streams.
The {\em IDLE} code is used when the Avalon-ST byte-stream
is unidirectional, eg., when the host is sending data to the
device and the device has no response data, the device sends
{\em IDLE} codes, and when the host needs to receive a
device response and has no data to send, the host sends
{\em IDLE} codes.

The JTAG-to-Avalon-ST interface is designed to transport binary
byte-streams, where every possible byte value can be transmitted.
Because the {\em IDLE} code uses a byte value, an {\em ESCAPE} 
code is needed. The {\em ESCAPE} code is used as an indicator 
within the byte-stream that a protocol code is being transmitted as data. 
The binary data value could simply be transmitted following the
{\em ESCAPE} code, however, Altera's protocol uses a data value
that is the logical exclusive or (XOR) of the data value with
\verb+0x20+ (this data masking technique is also
used in RFC1662\cite{RFC1662_1994}).
%
For example, a JTAG-to-Avalon-ST transaction containing the {\em IDLE}
code as data will encode the byte as \verb+0x4D 0x6A+,
while a transaction containing the {\em ESCAPE} code as data will
encode the byte as \verb+0x4D 0x6D+.

The data headers, and {\em IDLE} and {\em ESCAPE} codes appear in the
byte-streams on the {\em JTAG side} of the JTAG-to-Avalon-ST bridge.
The byte-streams on the {\em Avalon-ST side} of the JTAG-to-Avalon-ST bridge
contain {\em only} the data bytes. Section~\ref{sec:hardware}
shows simulation and logic analyzer traces of the byte-streams
on the JTAG-to-Avalon-ST interfaces.

% -----------------------------------------------------------------
% JTAG-to-Avalon-ST Instruction Register Codes
% -----------------------------------------------------------------
%
\begin{table}[p]
\caption{JTAG-to-Avalon-ST JTAG 3-bit Virtual Instruction Register (VIR) modes.}
\label{tab:jtag_to_avalon_st_vir_codes}
\begin{center}
\begin{tabular}{|c|l|p{10cm}|}
\hline
Code & Mode & Description\\
\hline\hline
&&\\
\verb+0+ & {\em DATA}     & Data byte-stream to and from user logic.\\
&&\\
\verb+1+ & {\em LOOPBACK} & Loopback the JTAG serial data (via a 1-bit bypass register).\\
&&\\
\verb+2+ & {\em DEBUG}    & 3-bit (read-only) Debug status register:\\
         &                & [0]: Avalon-ST reset state\\
         &                & [1]: Avalon-ST clock divided-by-2 state\\
         &                & [2]: Avalon-ST clock state\\
&&\\
\verb+3+ & {\em INFO}     & 11-bit (read-only) Component identification (Verilog parameter values).\\
         &                & [3:0]: $\floor(\log_2(\text{\tt UPSTREAM\_FIFO\_SIZE}))$\\
         &                & [7:4]: $\floor(\log_2(\text{\tt DOWNSTREAM\_FIFO\_SIZE}))$\\
         &                & [10:8]: {\tt PURPOSE} (0 = JTAG Avalon-ST, 1 = Avalon-MM)\\
&&\\
\verb+4+ & {\em CONTROL}  & 9-bit (read/write) Offset and reset-request control.\\
         &                & [7:0]: Offset\\
         &                & [9]: Reset request\\
&&\\
\hline
\end{tabular}
\end{center}
\end{table}

% -----------------------------------------------------------------
% JTAG-to-Avalon-ST DATA Mode Data Headers
% -----------------------------------------------------------------
%
\begin{table}[p]
\caption{JTAG-to-Avalon-ST DATA mode 16-bit headers.}
\label{tab:jtag_to_avalon_st_headers}
\begin{center}
\begin{tabular}{|l|l|p{9.5cm}|}
\hline
Data direction & Bits & Description\\
\hline\hline
&&\\
To device & [9:0]   & Scan length.\\
          & [12:10] & Read data length.\\
          & [15:13] & Write data length.\\
&&\\
From device & [0] & Read data is available.\\
&&\\
\hline
\end{tabular}
\end{center}
\end{table}

% -----------------------------------------------------------------
% JTAG-to-Avalon-ST DATA Mode Data Encoding Codes
% -----------------------------------------------------------------
%
\begin{table}[p]
\caption{JTAG-to-Avalon-ST DATA mode protocol codes.}
\label{tab:jtag_to_avalon_st_protocol_codes}
\begin{center}
\begin{tabular}{|c|l|p{10.5cm}|}
\hline
Code & Name & Description\\
\hline\hline
&&\\
\verb+0x4A+ & {\em IDLE}   & Inserted into the data stream when there is no data to send.\\
&&\\
\verb+0x4D+ & {\em ESCAPE} & Inserted into the data stream when the data to send is a protocol
               code, followed by the data XORed with \verb+0x20+.\\
&&\\
\hline
\end{tabular}
\end{center}
\end{table}

\clearpage
% -----------------------------------------------------------------
\subsection{JTAG-to-Avalon-MM}
% -----------------------------------------------------------------
\label{sec:jtag_to_avalon_mm}

% -----------------------------------------------------------------
% JTAG-to-Avalon-MM Bridge
% -----------------------------------------------------------------
%
\begin{figure}
  \begin{center}
    \includegraphics[width=\textwidth]{figures/jtag_to_avalon_mm_bridge.pdf}
  \end{center}
  \caption{JTAG-to-Avalon-MM Bridge.}
  \label{fig:jtag_to_avalon_mm_bridge}
\end{figure}
% -----------------------------------------------------------------

The JTAG-to-Avalon-MM bridge provides an interface for generating
Avalon-MM master read or write transactions via JTAG. The bridge
is implemented via a series of interconnected components and protocols.
Figure~\ref{fig:jtag_to_avalon_mm_bridge} shows the components within
the JTAG-to-Avalon-MM bridge, and how the bridge connects to an Avalon-MM system.
The host generates an Avalon-MM master transaction by first encoding
a Avalon-MM command into a command byte-stream, and then issuing the
command to the JTAG-to-Avalon-MM bridge via JTAG.
The {\em JTAG-to-Avalon-ST bridge} transports the encoded Avalon-ST command 
byte-stream (crossing from the JTAG clock domain to the Avalon-ST clock
domain), the {\em bytes-to-packets component} converts the command 
byte-stream into Avalon-ST command packet (which is a byte-stream 
with extra Avalon-ST control signals), the {\em packets-to-transactions}
component converts the Avalon-ST packet into an Avalon-MM
transaction, issues an Avalon-MM master transaction, and
encodes the response as an Avalon-ST response packet, the
{\em packets-to-bytes} component converts the Avalon-ST response packet
to an encoded Avalon-ST response byte-stream, which is then read
by the host via the {\em JTAG-to-Avalon-ST bridge}
(after the byte-stream has crossed from the Avalon-ST clock
domain to the JTAG clock domain).
The host decodes the response bytes to determine the completion of
an Avalon-MM write, or to access the data from an Avalon-MM read.

The Avalon-ST packets-to-transactions packet format for read
and write transactions is shown in 
Tables~\ref{tab:jtag_to_avalon_mm_read_packet}
and Table~\ref{tab:jtag_to_avalon_mm_write_packet}.
The transaction codes used in those packets are shown
in Table~\ref{tab:jtag_to_avalon_mm_transaction_codes}.
The transaction packets are encoded to and from byte-streams
via the bytes-to-packets/packets-to-bytes protocol
codes in Table~\ref{tab:jtag_to_avalon_mm_protocol_codes}.
The bytes-to-packets/packets-to-bytes protocol defines a channel 
number, start and end of packets, and an escape code. These
codes are consumed by the bytes-to-packets core and produced
by the packets-to-bytes core. These cores convert between
Avalon-ST {\em byte} streams and Avalon-ST {\em packet} streams.
The packets-to-transactions core converts the Avalon-ST packet
stream into an Avalon-MM transaction, and generates a response
Avalon-ST packet stream, which is subsequently encoded as an
Avalon-ST byte-stream.


The JTAG-to-Avalon-MM bridge is typically used for system debug
and initial board bring-up. The bridge can be used to generate
Avalon-MM single-read and single-write transactions. The bridge
can not be used to generate Avalon-MM bursts or to generate back-to-back 
transactions, however, the bridge could be used to program another
Avalon-MM master capable of generating bursts and back-to-back
transactions.

\clearpage
% -----------------------------------------------------------------
% Read Transaction
% -----------------------------------------------------------------
%
\begin{table}[p]
\caption{JTAG-to-Avalon-MM Packets-to-Transaction Read Packet.}
\label{tab:jtag_to_avalon_mm_read_packet}
\begin{center}
\begin{tabular}{|c|l|p{10cm}|}
\hline
Byte & Field & Description\\
\hline\hline
\multicolumn{3}{|l|}{\bf Command}\\
\hline
&&\\
\verb+0+ & Transaction code  & See Table~\ref{tab:jtag_to_avalon_mm_transaction_codes}\\
&&\\
\verb+1+ & Reserved          & Reserved for future use (use \verb+0x00+).\\
&&\\
\verb+[3:2]+ & Size          & Transaction size in bytes (16-bit big-endian format).\\
&&\\
\verb+[7:4]+ & Address       & Avalon-MM address (32-bit big-endian format).\\
&&\\
\hline
\multicolumn{3}{|l|}{\bf Response}\\
\hline
&&\\
\verb+[3:0]+ & Data          & Avalon-MM data (32-bit little-endian format).\\
&&\\
\hline
\end{tabular}
\end{center}
\end{table}

% -----------------------------------------------------------------
% Write Transaction
% -----------------------------------------------------------------
%
\begin{table}[p]
\caption{JTAG-to-Avalon-MM Packets-to-Transaction Write Packet.}
\label{tab:jtag_to_avalon_mm_write_packet}
\begin{center}
\begin{tabular}{|c|l|p{10cm}|}
\hline
Byte & Field & Description\\
\hline\hline
\multicolumn{3}{|l|}{\bf Command}\\
\hline
&&\\
\verb+0+ & Transaction code  & See Table~\ref{tab:jtag_to_avalon_mm_transaction_codes}\\
&&\\
\verb+1+ & Reserved          & Reserved for future use (use \verb+0x00+).\\
&&\\
\verb+[3:2]+ & Size          & Transaction size in bytes (16-bit big-endian format).\\
&&\\
\verb+[7:4]+ & Address       & Avalon-MM address (32-bit big-endian format).\\
&&\\
\verb+[n:8]+ & Data          & Avalon-MM data (32-bit little-endian format).\\
&&\\
\hline
\multicolumn{3}{|l|}{\bf Response}\\
\hline
&&\\
\verb+0+ & Transaction code  & Transaction code with the MSB inverted.\\
&&\\
\verb+1+ & Reserved          & Reserved for future use (use \verb+0x00+).\\
&&\\
\verb+[3:2]+ & Size          & Total number of bytes written (16-bit big-endian format).\\
&&\\
\hline
\end{tabular}
\end{center}
\end{table}

\clearpage
% -----------------------------------------------------------------
% Packets-to-transactions transaction codes
% -----------------------------------------------------------------
%
\begin{table}[p]
\caption{JTAG-to-Avalon-MM Packets-to-Transaction Transaction Codes.}
\label{tab:jtag_to_avalon_mm_transaction_codes}
\begin{center}
\begin{tabular}{|c|l|p{9cm}|}
\hline
Code & Avalon-MM Transaction & Description\\
\hline\hline
\multicolumn{3}{|l|}{\bf Transaction}\\
\hline
&&\\
\verb+0x00+ & Write, non-incrementing  & Write transaction(s) to a fixed address.\\
&&\\
\verb+0x04+ & Write, incrementing      & Write transaction(s) starting at the given address.\\
&&\\
\verb+0x10+ & Read, non-incrementing   & Read transaction(s) to a fixed address.\\
&&\\
\verb+0x14+ & Read, incrementing       & Read transaction(s) starting at the given address.\\
&&\\
\verb+0x7F+ & No transaction           & No transaction is generated on the Avalon-MM interface. Used for testing the packets interface.\\
&&\\
\hline
\end{tabular}
\end{center}
\end{table}

% -----------------------------------------------------------------
% Bytes-to-packets protocol codes
% -----------------------------------------------------------------
%
\begin{table}[p]
\caption{JTAG-to-Avalon-MM Bytes-to-Packets/Packets-to-Bytes Protocol Codes.}
\label{tab:jtag_to_avalon_mm_protocol_codes}
\begin{center}
\begin{tabular}{|c|l|p{10cm}|}
\hline
Code & Name & Description\\
\hline\hline
&&\\
\verb+0x7A+ & {\em SOP}     & Start of packet.\\
&&\\
\verb+0x7B+ & {\em EOP}     & End of packet.\\
&&\\
\verb+0x7C+ & {\em CHANNEL} & Channel number.\\
&&\\
\verb+0x7D+ & {\em ESCAPE}  & Inserted into the data stream when the
                              data to send is a protocol code, followed
                              by the data XORed with \verb+0x20+.\\
&&\\
\hline
\end{tabular}
\end{center}
\end{table}

\clearpage
% -----------------------------------------------------------------
\subsection{Altera Documentation}
% -----------------------------------------------------------------

The JTAG-to-Avalon-ST, bytes-to-packets, packets-to-bytes, and
JTAG-to-Avalon-MM components are (partially) documented in the 
Embedded IP Users Guide~\cite{Altera_Embedded_IP_2011};
%
\begin{itemize}
\item {\em Chapter 11: Avalon-ST Serial Periperal Interface Core}
has the documentation for an SPI controller. This is not used in the
JTAG-to-Avalon-MM design, but it is the only place in the IP guide that
the special character codes for the JTAG interface are discussed.
Page 11-2 has the special codes, i.e., \verb+0x4A+ ({\em IDLE}),
and \verb+0x4D+ ({\em ESCAPE}), and that the escape mask (XOR) code 
is \verb+0x20+.
\item {\em Chapter 18: SPI Slave/JTAG to Avalon Master Bridge Cores} 
describes the JTAG-to-Avalon-MM master component. Figure 18-3 contains an example of
a transaction byte stream, however, the channel code and start-of-packet are
reversed from what is observed in hardware (via SignalTap II traces).
There is no description of the figure in this chapter (hence the origin
of the \verb+0x4D+ escape code is not obvious).
\item {\em Chapter 20: Avalon-ST Bytes to Packets and Packets to Bytes Converter Cores}
describes the byte stream to transaction command packets components.
Page 28-2 describes the special codes used by the packets cores,
i.e., \verb+0x7A+ (start-of-packet, {\em SOP}), \verb+0x7B+ (end-of-packet,
{\em EOP}), \verb+0x7C+ ({\em CHANNEL}), \verb+0x7D+ ({\em ESCAPE}),
and that the escape mask (XOR) code is \verb+0x20+. Figure 20-3 has an example
of a transaction byte stream, and this time the channel code and start-of-packet
are in the order observed in hardware.
\item {\em Chapter 21: Avalon Packets to Transactions Converter Core}
describes the logic that generates an Avalon-MM transaction based on
a command packet, and then sends a response packet back via the byte stream.
Table 21-2 and 21-3 have the details required to form transaction packets
for Avalon-MM read and write operations.
Table 21-2 has the response packet format for write, but there is no documentation
for the read response packet (it was determined using SignalTap II
and the simulation testbench).
\item {\em Chapter 32: Avalon-ST JTAG Interface Core} describes the JTAG to
byte stream component. There are no comments regarding the protocol 
codes used for idle and escape, or the escape XOR mask.
This information used to be in this chapter; in the July 2010 version
of the guide, this component was described in Chapter 31, and there
was a table with the special character codes, and a description of how
the codes are used, along with the XOR mask for escaped data.
\end{itemize}

\clearpage
% =================================================================
\section{Software}
% =================================================================
\label{sec:software}

Altera provides two software interfaces for accessing JTAG
devices; SystemConsole and \verb+quartus_stp+. Altera provides
procedures for accessing the JTAG-to-Avalon components
using SystemConsole, but does not provide procedures for
use with \verb+quartus_stp+.
The code associated with this document contains a Tcl package
that provides \verb+quartus_stp+ support procedures for
accessing JTAG-to-Avalon components.

% -----------------------------------------------------------------
\subsection{SystemConsole}
% -----------------------------------------------------------------
\label{sec:systemconsole}

The {\em SystemConsole} services and procedures for accessing
JTAG components are documented in the {\em Quartus II Handbook, 
Volume 3: Verification, Chapter 10: Analyzing and Debugging 
Designs with the System Console}~\cite{Altera_Quartus_Handbook_2011}.
The services of interest are;
%
\begin{itemize}
\item \verb+sld+: provides low-level access to the Virtual JTAG
instruction and data registers.
%
\item \verb+jtag_debug+: provides procedures to access to the status and
control registers in the JTAG-to-Avalon-ST bridge.
%
\item \verb+bytestream+: provides procedures to send and
receive byte-streams via the data mode of the JTAG-to-Avalon-ST bridge.
%
\item \verb+master+: provides procedures to generate Avalon-MM
read and write transactions using the JTAG-to-Avalon-MM bridge.
%
\end{itemize}
%
Using the protocols provided in Section~\ref{sec:protocol}, the
\verb+master+ service procedures can be re-implemented using the
\verb+bytestream+ service procedures, and all service procedures
can be implemented using the \verb+sld+ service procedures.

% -----------------------------------------------------------------
\subsection{Command-line tools ({\tt quartus\_stp})}
% -----------------------------------------------------------------
\label{sec:quartus_stp}

The \verb+quartus_stp+ procedures for accessing JTAG components
are documented in the {\em Virtual JTAG Megafunction User 
Guide}~\cite{Altera_Virtual_JTAG_2008}.
The procedures of interest are;
%
\begin{itemize}
\item \verb+device_ir_shift+ and \verb+device_dr_shift+: provide
low-level access to the JTAG instruction and data registers.
%
\item \verb+device_virtual_ir_shift+ and \verb+device_virtual_dr_shift+: provide
low-level access to the Virtual JTAG instruction and data registers.
\end{itemize}
%
The Virtual JTAG functions in \verb+quartus_stp+ are essentially
identical to those implemented by the \verb+sld+ service
(with the procedure names changed). The \verb+quartus_stp+ Virtual
JTAG procedures can be implemented using the low-level
JTAG procedures. Understanding the implementation of the
JTAG-to-Avalon-MM protocol via the low-level JTAG commands
allows the development of custom software to access Avalon-MM
components via JTAG.

\clearpage
% =================================================================
\section{Hardware Tests}
% =================================================================
\label{sec:hardware}

The source code to the JTAG-to-Avalon-ST and JTAG-to-Avalon-MM
components is provided as part of the Quartus II software
installation as shown in Figure~\ref{fig:quartus_jtag_source}.
%
Host software accesses the JTAG-to-Avalon-ST byte-streams and
generates JTAG-to-Avalon-MM transactions via Tcl procedures
provided by the {\em SystemConsole} user interface. The source
code for the SystemConsole procedures is not provided
in the Quartus II software installation.

The hardware tests in this section determine how SystemConsole
utilizes the JTAG-to-Avalon protocols. 
Each test instantiates a JTAG-to-Avalon component and a SignalTap
II logic analyzer instance, issues a SystemConsole command to
the hardware, and captures a logic analyzer trace of the JTAG
transaction.
The logic analyzer traces are compared to the protocols in
Section~\ref{sec:protocol} to determine the {\em software 
implementation} of the protocols.
The JTAG transactions are then reproduced in simulation.
The result of this analysis is that the JTAG-to-Avalon-MM bridge
can be used in simulation to generate Avalon-MM transactions
(Altera do not officially provide such support).

The JTAG-to-Avalon-ST and JTAG-to-Avalon-MM components are typically
instantiated in Qsys~\cite{Altera_Qsys_Documentation} or in SOPC 
Builder\cite{Altera_SOPC_Builder_Documentation}.
%
The hardware designs in this section instantiate the
JTAG components directly. Refer to the simulation and synthesis
scripts for details.

The hardware tests were performed using an Arrow BeMicro-SDK
USB stick. The USB stick contains a USB-Blaster interface and
a Cyclone IV FPGA. The hardware tests make minimal use of the
board I/O, using only the clock, reset, 8 LEDs, and 2 DIP switches.
The hardware tests can be ported to other development boards.

% -----------------------------------------------------------------
% Quartus JTAG Source
% -----------------------------------------------------------------
%
\begin{figure}[p]
\begin{minipage}{\textwidth}
\begin{itemize}
\item \verb+QUARTUS_ROOTDIR+

Environment variable created by the Quartus II installer.
%
\item \verb+$QUARTUS_ROOTDIR/eda/sim_lib/altera_mf.v+

Altera Megafunction Verilog file containing the \verb+sld_virtual_jtag+ source.
%
\item \verb+$QUARTUS_ROOTDIR/eda/sim_lib/altera_mf.vhd+ 

Altera Megafunction VHDL file containing the \verb+sld_virtual_jtag+ source.
%
\item \verb+$QUARTUS_ROOTDIR/eda/sim_lib/altera_mf_components.vhd+ 

Altera Megafunction VHDL file containing the \verb+sld_virtual_jtag+ and\\ \verb+sld_virtual_jtag_basic+ component definitions.
%
\item \verb+QUARTUS_SOPC_IP = $QUARTUS_ROOTDIR/../ip/altera/sopc_builder_ip+

Quartus SOPC Builder IP directory.
%
\item \verb+$QUARTUS_SOPC_IP/altera_avalon_jtag_phy/altera_jtag_sld_node.v+ 

The JTAG node. 

For {\em synthesis} this component instantiates
an \verb+sld_virtual_jtag_basic+ component, which is
an \verb+sld_virtual_jtag+ component with additional parameters to set the
manufacturer ID (110 = 0x6E), type ID (132 = 0x84), and version (1). 
There is no source for the Verilog or VHDL component, just a VHDL 
component definition.

For {\em simulation} this component provide a series of Verilog tasks
that are used to generate JTAG transactions.
%
\item \verb+$QUARTUS_SOPC_IP/altera_avalon_jtag_phy/altera_jtag_streaming.v+ 

JTAG-to-Avalon-ST protocol component, including the 3-bit JTAG instruction
register (IR) values (see Table~\ref{tab:jtag_to_avalon_st_ir_codes}),
 and byte stream header and data format details
(not exactly easy reading, but enough detail to write simulation stimulus
to aid in understanding). 
%
\item \verb+$QUARTUS_SOPC_IP/altera_avalon_jtag_phy/altera_avalon_st_jtag_interface.v+ 

JTAG-to-Avalon-ST component; instantiates the JTAG node and streaming components.
%
\item \verb+$QUARTUS_SOPC_IP/altera_avalon_st_bytes_to_packets/+ 

\verb+$QUARTUS_SOPC_IP/altera_avalon_st_packets_to_bytes/+ 

Avalon-ST bytes-to-packets and packets-to-bytes components which
implement the byte encoding and decoding protocol in
Table~\ref{tab:jtag_to_avalon_mm_protocol_codes}.

\item \verb+$QUARTUS_SOPC_IP/altera_avalon_packets_to_master/altera_avalon_packets_to_master.v+ 

Avalon-ST packets to Avalon-MM master transactions component, which
implements the transaction codes in 
Table~\ref{tab:jtag_to_avalon_mm_transaction_codes}, the 
write packet protocol in Table~\ref{tab:jtag_to_avalon_mm_write_packet}, and the
read packet protocol in Table~\ref{tab:jtag_to_avalon_mm_read_packet}.
%
\item \verb+$QUARTUS_SOPC_IP/altera_jtag_avalon_master/altera_jtag_avalon_master.v+ 

JTAG-to-Avalon-MM component; instantiates the JTAG node, streaming, bytes-to-packets,
packets-to-bytes, and packets-to-transactions components.
%
\end{itemize}
\end{minipage}
\caption{Quartus II JTAG-to-Avalon source.}
\label{fig:quartus_jtag_source}
\end{figure}

\clearpage
% -----------------------------------------------------------------
\subsection{JTAG Node}
% -----------------------------------------------------------------

% -----------------------------------------------------------------
% JTAG node project source
% -----------------------------------------------------------------
%
\begin{table}[t]
\caption{JTAG node ({\tt jtag\_node}) project source.}
\label{tab:jtag_node_source}
\begin{center}
\begin{tabular}{|l|l|p{7cm}|}
\hline
Folder & File & Description\\
\hline\hline
&&\\
\verb+scripts+ & \verb+synth.tc+         & Quartus II synthesis script\\
               & \verb+bemicro_sdk.stp+  & SignalTap II setup\\
               & \verb+bemicro_sdk.sdc+  & Timing constraints\\
               & \verb+sim.tc+           & Modelsim simulation script\\
               & \verb+jtag_node_tb.do+  & Modelsim wave window setup\\
               & \verb+jtag_cmds_sc.tcl+ & SystemConsole commands\\
&&\\
\verb+src+     & \verb+bemicro_sdk.sv+ & Top-level synthesis source\\
&&\\
\verb+test+    & \verb+jtag_node_tb.sv+ & Simulation testbench\\
&&\\
\hline
\end{tabular}
\end{center}
\end{table}
% -----------------------------------------------------------------

% -----------------------------------------------------------------
% JTAG node address map
% -----------------------------------------------------------------
%
\begin{table}[t]
\caption{JTAG node ({\tt jtag\_node}) project Virtual Instruction Register (VIR) decode.}
\label{tab:jtag_node_registers}
\begin{center}
\begin{tabular}{|c|c|p{7cm}|}
\hline
VIR[2:0] & Access & Description\\
\hline\hline
&&\\
0   & Write-only & Write to the 8-bit data register\\
1   & Read-only  & Read from the 8-bit data register\\
2   & Read-only  & Read the state of the 2 DIP switches\\
3-7 & Read-only  & Unused. Reads return zero\\
&&\\
\hline
\end{tabular}
\end{center}
\end{table}
% -----------------------------------------------------------------

The JTAG node source, \verb+altera_jtag_sld_node.v+ (see 
Figure~\ref{fig:quartus_jtag_source} for the source location),
provides the basic JTAG-to-FPGA interface logic. The component
selects between the Virtual JTAG component for {\em synthesis}
and a set of Verilog tasks for {\em simulation}. 
The hardware tests on the JTAG node determine the JTAG transaction
sequences. These sequences are reproduced (as closely as possible)
via the JTAG node simulation tasks. The JTAG node simulation model
is then used to develop the JTAG-to-Avalon-ST and JTAG-to-Avalon-MM
simulation models.

The JTAG node project source layout is shown in Table~\ref{tab:jtag_node_source}.
Appendix~\ref{app:build} contains build instructions.
The top-level synthesis source contains the JTAG node, an 8-bit JTAG
data shift-register, and the registers shown in
Table~\ref{tab:jtag_node_registers}. Write and read access to the 8-bit
data register uses two different VIR codes, due to the fact that JTAG
transactions are {\em simultaneously} both write (host-to-device data)
and read (device-to-host data) transactions; when VIR = 0, the
device transmits zero over JTAG (and the host ignores it),
and when VIR = 1, the host transmits zero over JTAG (and the
device ignores it). This simple design is representative of how
all Virtual JTAG designs use the Virtual Instruction Register for
decoding different Virtual Data Register data paths.

\newpage
The JTAG node design uses the LEDs for user feedback as follows;
%
\begin{center}
\begin{tabular}{ll}
LED[3:0] & Connects to the 4-LSBs of the 8-bit data register\\
LED[6:4] & The 3-bit VIR value\\
LED[7]   & Blinks at about 1Hz to show the design is running\\
\end{tabular}
\end{center}
%
The Tcl procedure \verb+jtag_sld_vir_count+ generates an incrementing
count on the 3 VIR LEDs (and reads the 2-bit switch state via VIR out),
while the procedure \verb+jtag_sld_data_count+
generates a count on the 4 data LEDs (and reads the 8-bit register
back and reads the 2-bit switch state). The value written to the
LED and the values read back are printed to the console. The LED 
count increments at about once per second, so the DIP switches can
be changed and the value printed to the console seen to change.

Figures~\ref{fig:jtag_node_vir} and~\ref{fig:jtag_node_vdr}
show SignalTap II logic analyzer traces from the two basic Virtual
JTAG transactions; a Virtual shift-IR sequence and a Virtual
shift-DR sequence. The figures show the JTAG one-hot states
as well as the Virtual JTAG one-hot states. The Virtual sequences
are constructed from a JTAG shift-IR sequence to load the
\verb+USER1+ or \verb+USER0+ JTAG instruction codes, and then
a JTAG shift-DR sequence to access the Virtual IR or Virtual DR
path. The Virtual IR sequence in Figure~\ref{fig:jtag_node_vir} 
was generated by the Tcl command \verb+jtag_sld_data_read+,
which sets IR = 1 (per Table~\ref{tab:jtag_node_registers}),
while the Virtual DR sequence in  Figure~\ref{fig:jtag_node_vdr} 
was generated by the Tcl command \verb+jtag_sld_data_write 0x55+,
which sets IR = 0 (per Table~\ref{tab:jtag_node_registers})
prior to the trace capture, and then serially shifts in the
8-bit data value \verb+0x55+ (overwriting the previous register value
of \verb+0xAA+).
For more detail on the Virtual JTAG interface 
see~\cite{Hawkins_VJTAG_2011} (that document explains the
encoding of the 10-bit VIR code in Figure~\ref{fig:jtag_node_vir}).


The JTAG node source, \verb+altera_jtag_sld_node.v+ provides a
minimal set of Verilog tasks for simulation. The project
testbench, \verb+jtag_node_tb.sv+, contains test sequences
showing how to use the JTAG node Verilog tasks to update the
Virtual Instruction Register to generate the VIR codes shown
in Table~\ref{tab:jtag_to_avalon_st_vir_codes}, and how to
generate Virtual shift-DR sequences for a single-byte and
for multiple-bytes. The simulation {\em does not} generate
waveforms identical to the hardware tests (due to the limitations
of the Verilog tasks), but the functionality is sufficient to
simulate designs containing the JTAG node.

% -----------------------------------------------------------------
% SignalTap II logic analyzer traces
% -----------------------------------------------------------------
%
% The traces were created using;
% * Shift-Ctrl-PrintScreen
% * Crop using Corel PhotoPaint
% * Convert to bmp and back to png using png2bmp and bmp2png.
%   If you do not do this, you get an exception from PCTex.
% * Paste into Inkscape without scaling the bitmap.
% * Add overlay text.
%
% The notes in my vjtag.tex document, show how to crop using
% MATLAB.
%
\begin{landscape}
\begin{figure}
  \centering
  \includegraphics[width=210mm]
  {figures/jtag_node_vir.pdf}
  \caption{JTAG node Virtual Instruction Register SignalTap II logic
  analyzer trace. The sequence consists of the JTAG {\tt USER1} 
  IR-shift sequence followed by a JTAG DR-shift sequence.
  The JTAG shift-DR sequence corresponds to the Virtual JTAG 
  shift-IR sequence. In this example, the sequence sets the
  Virtual Instruction to {\tt ir\_in[2:0] = 1}.}
  \label{fig:jtag_node_vir}
\end{figure}
\end{landscape}

\begin{landscape}
\begin{figure}
  \centering
  \includegraphics[width=210mm]
  {figures/jtag_node_vdr.pdf}
  \caption{JTAG node Virtual Data Register SignalTap II logic
  analyzer trace. The sequence consists of the JTAG {\tt USER0} 
  IR-shift sequence followed by a JTAG DR-shift sequence.
  The JTAG shift-DR sequence corresponds to the Virtual JTAG 
  shift-DR sequence. In this example, the sequence sets
  the 8-bit data register to {\tt data[7:0] = 0x55}.}
  \label{fig:jtag_node_vdr}
\end{figure}
\end{landscape}


\clearpage
% -----------------------------------------------------------------
\subsection{JTAG-to-Avalon-ST}
% -----------------------------------------------------------------
% -----------------------------------------------------------------
% JTAG-to-Avalon-ST project source
% -----------------------------------------------------------------
%
\begin{table}[t]
\caption{JTAG-to-Avalon-ST ({\tt jtag\_to\_avalon\_st}) project source.}
\label{tab:jtag_to_avalon_st_source}
\begin{center}
\begin{tabular}{|l|l|p{6cm}|}
\hline
Folder & File & Description\\
\hline\hline
&&\\
\verb+scripts+ & \verb+synth.tc+         & Quartus II synthesis script\\
               & \verb+bemicro_sdk.stp+  & SignalTap II setup\\
               & \verb+bemicro_sdk.sdc+  & Timing constraints\\
               & \verb+sim.tc+           & Modelsim simulation script\\
               & \verb+jtag_to_avalon_st_tb.do+  & Modelsim wave window setup\\
               & \verb+jtag_cmds_sc.tcl+ & SystemConsole commands\\
&&\\
\verb+src+     & \verb+bemicro_sdk.sv+ & Top-level synthesis source\\
&&\\
\verb+test+    & \verb+jtag_to_avalon_st_tb.sv+ & Simulation testbench\\
&&\\
\hline
\end{tabular}
\end{center}
\end{table}
% -----------------------------------------------------------------

The JTAG-to-Avalon-ST source, \verb+altera_avalon_st_jtag_interface.v+ (see 
Figure~\ref{fig:quartus_jtag_source} for the source location),
provides an interface for the host to stream bytes into and out of
the FPGA fabric, and to access control and status registers.
The JTAG-to-Avalon-ST protocol is described in 
Section~\ref{sec:jtag_to_avalon_st}, and the SystemConsole procedures
that implement the protocol are referenced in 
Section~\ref{sec:systemconsole}.

The JTAG-to-Avalon-ST project source layout is shown in 
Table~\ref{tab:jtag_to_avalon_st_source}.
Appendix~\ref{app:build} contains build instructions.
The top-level synthesis source contains the JTAG-to-Avalon-ST bridge,
an 8-bit data register (which is loaded each time valid data is received
on the host-to-device Avalon-ST byte-stream), a write data counter
(to track writes to the data register), and
a synthesis parameter to control whether the Avalon-ST interface
is looped-back (allowing the host to read-back the data it writes)
or whether host-to-device data is just dropped (and no device-to-host
data is ever generated).
The JTAG-to-Avalon-ST design uses the LEDs for user feedback as follows;
%
\begin{center}
\begin{tabular}{ll}
LED[5:0] & Connects to the 6-LSBs of the 8-bit data register\\
LED[6]   & The Avalon-ST \verb+resetrequest+ output\\
LED[7]   & Blinks at about 1Hz to show the design is running\\
\end{tabular}
\end{center}
%
The SystemConsole Tcl procedures in \verb+jtag_cmds_sc.tcl+ allow
the data register (and LEDs) to be written and read, allow the
transmission of strings and blocks of binary data over the byte-stream
interface, control of the \verb+resetrequest+ output, and
implementation of some of the \verb+bytestream+ service commands
via the low-level \verb+sld+ service commands. See the Tcl
script for details.

The JTAG-to-Avalon-ST source file, \verb+altera_jtag_streaming.v+,
requires the following modifications to enable SignalTap II
probing (to disable logic elimination and net renaming);
%
\begin{verbatim}
     wire write_data_byte_aligned          /* synthesis keep */;
     reg [15:0] header_in  = 'b0           /* synthesis noprune */;
     wire [7:0] idle_inserter_source_data  /* synthesis keep */;
     wire [7:0] idle_inserter_source_data;
     wire data_available                   /* synthesis keep */;
\end{verbatim}
%

Figure~\ref{fig:jtag_to_avalon_st_hello} shows a SignalTap II logic
analyzer trace of the transmission of the string ``hello''. The
string was transmitted by the \verb+jtag_bytestream_string+
procedure in the script \verb+jtag_cmds_sc.tcl+. The
procedure is implemented using the SystemConsole \verb+bytestream+
service, \verb+bytestream_send+ command. The trace, and
others like it, show how the \verb+bytestream+ service utilizes
the JTAG-to-Avalon-ST protocol.


Figure~\ref{fig:jtag_to_avalon_st_HIJKLMN} shows a SignalTap II logic
analyzer trace of the transmission of the string ``HIJKLMN''.
This string is special in that it contains the {\em IDLE}
(\verb+0x4A+) and {\em ESCAPE} (\verb+0x4D+) codes
as data. The logic analyzer trace shows how the data is encoded
over the JTAG byte-stream and that the data on the Avalon-ST
interface matches the bytes in the transmitted string.

Figures~\ref{fig:jtag_to_avalon_st_sop},
~\ref{fig:jtag_to_avalon_st_idle}, ~\ref{fig:jtag_to_avalon_st_eop}, 
and~\ref{fig:jtag_to_avalon_st_sld_eop}, show a 1kB transfer of
binary data over the JTAG-to-Avalon-ST interface.
The binary data consists of a start-of-packet (SOP) code ({\tt 0x11}),
alternating data bytes {\tt 0x55} and {\tt 0xAA}, and an
end-of-packet (EOP) code ({\tt 0x22}). The format of the binary data
was chosen to provide unique SOP and EOP codes to trigger 
SignalTap II, and avoided using any of the JTAG-to-Avalon-ST 
protocol codes (a data payload of an incrementing count
would have to avoid the protocol codes).
The figures show a subtle implementation detail of the
SystemConsole \verb+bytestream+ service, \verb+bytestream_send+
procedure; for each 1024-byte transfer only 1022-bytes of data
are transmitted, with the final two bytes replaced with the
{\em IDLE} code (see Figure~\ref{fig:jtag_to_avalon_st_idle}).
The final two bytes of data are then transmitted
in a subsequent data packet (see Figure~\ref{fig:jtag_to_avalon_st_eop}).
Figure~\ref{fig:jtag_to_avalon_st_sld_eop} shows that the 1kB of
data could have been transferred in a single 1kB transaction.

The JTAG-to-Avalon-ST SystemConsole \verb+bytestream+ procedures
were tested using the binary data for various lengths,
yielding the following observations;
%
\begin{itemize}
\item The JTAG-to-Avalon-ST header is always \verb+0xFC03+ which
indicates that the scan length, write length, and read length
are all 1024-bytes (0x400).
%
\item The 1024-byte length header is used for all transfers,
including transfers of greater than 1024-bytes; the SystemConsole
procedure divides the transfer into multiple 1024-byte transactions.
%
\end{itemize}
%
The JTAG-to-Avalon-ST protocol was implemented using the 
SystemConsole \verb+sld+ service for an arbitrary length
binary data stream. The JTAG-to-Avalon-ST header was calculated
as the next 256-byte increment over the binary data length,
with {\em IDLE} codes used as padding. Signal Tap II traces
of various lengths confirmed the operation of the procedures.

The \verb+sld+ based procedures also provided a mechanism to
exercise the data available field of the JTAG-to-Avalon-ST
read data header in Table~\ref{tab:jtag_to_avalon_st_headers}.
The data available signal, \verb+data_available+, can be
seen asserted at the end of the transaction in
Figure~\ref{fig:jtag_to_avalon_st_sld_eop}. The 16-bit
read data header following this transaction had the 
read data available bit set.

The project simulation testbench, \verb+jtag_to_avalon_st_tb.sv+,
uses the Verilog tasks in the JTAG SLD node to access the
JTAG-to-Avalon-ST bridge control and status registers,
and to generate data byte-stream transactions. The testbench
contains stimulus sequences that test different JTAG-to-Avalon-ST
headers, eg., read-only and write-only operations (modes that
are not exercised by any of the Altera procedures), and
stimulus that investigates what happens to the bridge interface
logic when the JTAG interface does not send the amount of data
indicated by the header, i.e., short transactions.

% -----------------------------------------------------------------
% SignalTap II logic analyzer traces
% -----------------------------------------------------------------
%
% The traces were created using;
% * Shift-Ctrl-PrintScreen
% * Crop using Corel PhotoPaint
% * Convert to bmp and back to png using png2bmp and bmp2png.
%   If you do not do this, you get an exception from PCTex.
% * Paste into Inkscape without scaling the bitmap.
% * Add overlay text.
%
% The notes in my vjtag.tex document, show how to crop using
% MATLAB.
%
\begin{landscape}
\begin{figure}
  \centering
  \includegraphics[width=210mm]
  {figures/jtag_to_avalon_st_hello.pdf}
  \caption{JTAG-to-Avalon-ST loopback of the string ``hello''.
  The 16-bit header indicates that the SystemConsole
  {\tt bytestream\_send} command generates a 1kB packet.
  The packet data consists of the characters in the string
  ``hello'', followed by the {\em IDLE} code ({\tt 0x4A}).}
  \label{fig:jtag_to_avalon_st_hello}
\end{figure}
\end{landscape}

\begin{landscape}
\begin{figure}
  \centering
  \includegraphics[width=210mm]
  {figures/jtag_to_avalon_st_HIJKLMN.pdf}
  \caption{JTAG-to-Avalon-ST loopback of the string ``HIJKLMN''.
  This string is special in that it contains the {\em IDLE}
  and {\em ESCAPE} character codes as {\em data}.
  The logic analyzer trace shows how the data is encoded
  in the JTAG byte-stream as an {\em ESCAPE} code followed
  by the byte XORed with {\tt 0x20}.}
  \label{fig:jtag_to_avalon_st_HIJKLMN}
\end{figure}
\end{landscape}

\begin{landscape}
\begin{figure}
  \centering
  \includegraphics[width=210mm]
  {figures/jtag_to_avalon_st_1kB_sop.pdf}
  \caption{JTAG-to-Avalon-ST transfer of a 1kB of binary data;
  logic analyzer trace at the start-of-packet.}
  \label{fig:jtag_to_avalon_st_sop}
\end{figure}
\end{landscape}

\begin{landscape}
\begin{figure}
  \centering
  \includegraphics[width=210mm]
  {figures/jtag_to_avalon_st_1kB_idle.pdf}
  \caption{JTAG-to-Avalon-ST transfer of a 1kB of binary data;
  logic analyzer trace at the end of the 1kB scan length.
  The SystemConsole {\tt bytestream\_send} procedure inserts two
  {\em IDLE} codes at the end of the packet, and sends the
  last two bytes in another transfer.}
  \label{fig:jtag_to_avalon_st_idle}
\end{figure}
\end{landscape}

\begin{landscape}
\begin{figure}
  \centering
  \includegraphics[width=210mm]
  {figures/jtag_to_avalon_st_1kB_eop.pdf}
  \caption{JTAG-to-Avalon-ST transfer of a 1kB of binary data;
  logic analyzer trace at the end-of-packet.
  The SystemConsole {\tt bytestream\_send} procedure inserts two
  {\em IDLE} codes at the end of the packet, and sends the
  last two bytes in another transfer.}
  \label{fig:jtag_to_avalon_st_eop}
\end{figure}
\end{landscape}

\begin{landscape}
\begin{figure}
  \centering
  \includegraphics[width=210mm]
  {figures/jtag_to_avalon_st_1kB_sld_eop.pdf}
  \caption{JTAG-to-Avalon-ST transfer of a 1kB of binary data;
  logic analyzer trace at the end-of-packet.
  The SystemConsole {\tt bytestream\_send} procedure was
  implemented using the low-level {\tt sld} service procedures
  to show that the full 1kB data transfer can be performed
  using a single 1kB transfer. Note that the {\tt data\_available}
  signal remains asserted at the end of the transaction, as
  Avalon-ST loopback data is ready to be read by the host.}
  \label{fig:jtag_to_avalon_st_sld_eop}
\end{figure}
\end{landscape}


\clearpage
% -----------------------------------------------------------------
\subsection{JTAG-to-Avalon-MM}
% -----------------------------------------------------------------
% -----------------------------------------------------------------
% JTAG-to-Avalon-MM project source
% -----------------------------------------------------------------
%
\begin{table}[t]
\caption{JTAG-to-Avalon-MM ({\tt jtag\_to\_avalon\_mm}) project source.}
\label{tab:jtag_to_avalon_mm_source}
\begin{center}
\begin{tabular}{|l|l|p{6cm}|}
\hline
Folder & File & Description\\
\hline\hline
&&\\
\verb+scripts+ & \verb+synth.tc+         & Quartus II synthesis script\\
               & \verb+bemicro_sdk.stp+  & SignalTap II setup\\
               & \verb+bemicro_sdk.sdc+  & Timing constraints\\
               & \verb+sim.tc+           & Modelsim simulation script\\
               & \verb+jtag_to_avalon_mm_tb.do+  & Modelsim wave window setup\\
               & \verb+jtag_cmds_sc.tcl+ & SystemConsole commands\\
&&\\
\verb+src+     & \verb+bemicro_sdk.sv+   & Top-level synthesis source\\
&&\\
\verb+test+    & \verb+jtag_to_avalon_mm_tb.sv+ & Simulation testbench\\
&&\\
\hline
\end{tabular}
\end{center}
\end{table}
% -----------------------------------------------------------------

The JTAG-to-Avalon-MM source, \verb+altera_jtag_avalon_master.v+ (see 
Figure~\ref{fig:quartus_jtag_source} for the source location),
provides an interface for the host to generate Avalon-MM master
transactions (and to access control and status registers of the
underlying JTAG-to-Avalon-ST component).
The JTAG-to-Avalon-MM protocol is described in 
Section~\ref{sec:jtag_to_avalon_mm}, and the SystemConsole procedures
that implement the protocol are referenced in 
Section~\ref{sec:systemconsole}.

The JTAG-to-Avalon-MM project source layout is shown in 
Table~\ref{tab:jtag_to_avalon_mm_source}.
Appendix~\ref{app:build} contains build instructions.
The top-level synthesis source contains the JTAG-to-Avalon-MM bridge,
and a 32-bit data register located at 32-bit aligned address
\verb+0x11223344+. The register address makes it easier to see
the big-endian ordering of the address bytes in the JTAG-to-Avalon-ST
byte-streams.
The JTAG-to-Avalon-MM design uses the LEDs for user feedback as follows;
%
\begin{center}
\begin{tabular}{ll}
LED[5:0] & Connects to the 6-LSBs of the 32-bit data register\\
LED[6]   & The Avalon-ST \verb+resetrequest+ output\\
LED[7]   & Blinks at about 1Hz to show the design is running\\
\end{tabular}
\end{center}
%
The SystemConsole Tcl procedures in \verb+jtag_cmds_sc.tcl+ allow
the data register (and LEDs) to be written and read, control of 
the \verb+resetrequest+ output, and implementation of some of 
the \verb+master+ service commands via the \verb+bytestream+
service commands. See the Tcl script for details.

The SystemConsole \verb+master+ service provides the
Tcl procedures shown in Table~\ref{tab:systemconsole_master_procedures}
(Quartus II Handbook, Volume 3, Chapter 10~\cite{Altera_Quartus_Handbook_2011}).
The following tests were performed to investigate the JTAG transactions
generated by the \verb+master+ service procedures;
%
\begin{itemize}
\item 32-bit write access via \verb+master_write_32+.

Figures~\ref{fig:jtag_to_avalon_mm_write_start} 
and~\ref{fig:jtag_to_avalon_mm_write_end} show a SignalTap II logic
analyzer trace of a 32-bit write transaction for 
(address, data) = (\verb+0x11223344+, \verb+0x55667788+).
The figures are annotated to show the encoded transaction bytes.

\item 32-bit read access via \verb+master_read_32+.

Figures~\ref{fig:jtag_to_avalon_mm_read_start} 
and~\ref{fig:jtag_to_avalon_mm_read_end} show a logic
analyzer trace of a 32-bit read transaction for 
(address, data) = (\verb+0x11223344+, \verb+0x55667788+).
The figures are annotated to show the encoded transaction bytes.

\item 32-bit write/read access via \verb+master_write/read_memory+.

Figures~\ref{fig:jtag_to_avalon_mm_write_memory} 
and~\ref{fig:jtag_to_avalon_mm_read_memory} show a logic
analyzer trace of multiple 32-bit write and read transactions.
The figures are annotated to show the encoded transaction bytes.
The key difference between \verb+master_write/read_memory+
and \verb+master_write/read_32+ procedures is that the JTAG-to-Avalon-ST
encoded byte-streams are used more efficiently for the
\verb+master_write/read_memory+ procedures.

\item Protocol codes as data.

Figure~\ref{fig:jtag_to_avalon_mm_write_0x7D7C7B7A} shows a logic
analyzer trace of a 32-bit write for 
(address, data) = (\verb+0x11223344+, \verb+0x7D7C7B7A+),
i.e., the write data contains the bytes-to-packets
protocol codes {\em as data}. The logic analyzer trace shows
the {\em ESCAPE} bytes and XORed data in the JTAG and
Avalon-ST byte-streams.
%
Figure~\ref{fig:jtag_to_avalon_mm_write_0x4D4A} shows a logic
analyzer trace of a 16-bit write for 
(address, data) = (\verb+0x11223344+, \verb+0x4D4A+),
i.e., the write data contains the Avalon-ST byte-stream
protocol codes {\em as data}. The logic analyzer trace shows
the {\em ESCAPE} bytes and XORed data in the JTAG byte-stream.

\item Address alignment.

The \verb+master_read/write_16/32+ procedures enforce 16-bit and
32-bit address alignment at the software interface, i.e., the 
procedures generate error messages if the address argument is
unaligned.

The \verb+master_read/write_memory+ procedures accept an arbitrary
address. For unaligned accesses the software generates a mixture
of 8-bit, 16-bit, and 32-bit Avalon-MM transactions (depending on
the alignment of the start address and the length of the data).

\item Number of Avalon-MM transactions per JTAG transaction.

The \verb+master_read/write_8/16/32+ procedures generate a
JTAG transaction {\em per data value}. Each JTAG
transaction consists of a 256-byte transfer;
the JTAG-to-Avalon-ST header is \verb+0xFC00 = 111_111_00_0000_0000b+
which indicates that the scan length, write length, and read length
are all 256-bytes. The JTAG transfer rate for multiple words is 
{\em slow}.

The \verb+master_read/write_memory+ procedures generate a
JTAG transaction for {\em all data values} if the address
is aligned and the data length is consistent, eg., a 32-bit
aligned address and a data length that is a multiple of
four bytes. If the address is not aligned, or the length is
not a multiple of four bytes, then several JTAG transactions
will be generated. The \verb+master_read/write_memory+
procedures should be used when transferring
large amounts of data over the JTAG interface, eg., filling
or reading RAM contents.

\item JTAG data transfer rate.

The performance over JTAG is determined by a combination of hardware 
and software. The USB-Blaster determines the performance at the
hardware layer. The USB-Blaster performs parallel-to-serial
conversion of host-to-device bytes from USB to the JTAG TDI input,
and then serial-to-parallel conversion of the JTAG TDI output
to bytes for transfer of device-to-host bytes over USB.
The USB-Blaster operates in two modes; bit-mode and byte-mode;
bit-mode is used when manipulating the JTAG TAP machine states,
and byte-mode is used when shifting data. The two modes can
be see in the SignalTap II traces; at the beginning of the
traces the USB-Blaster uses bit-mode and then switches to
byte-mode to transfer the bulk of the byte-stream. Note how
the JTAG clock (TCK) period is longer in bit-mode than in byte-mode.

The SignalTap II traces show that when the USB-Blaster is operating
in byte-mode, each byte is transfered using 8 clocks at 6MHz clock
period plus about two 6MHz periods of dead-time between each byte 
(where TCK is high), i.e., about 10
6MHz periods per byte, or $6\text{MHz}/(10\times1024) = 586$kB/s.

The measured performance of transfers from the software layer
(SystemConsole) for 4kB (0x1000), 8kB (0x2000), and 16kB (0x4000)
were;
\begin{itemize}
\item \verb+master_write_memory+: between 180kB/s and 280kB/s
\item \verb+master_read_memory+: between 80kB/s and 140kB/s
\end{itemize}
%
\item JTAG-to-Avalon-ST header.

The \verb+master_read/write_8/16/32+ procedures generate a
JTAG transaction for each data value, so the encoded byte-stream
for the host-to-device command and device-to-host response is
small, and hence the smallest JTAG-to-Avalon-ST transaction
can be used. The smallest JTAG-to-Avalon-ST transaction is 256-bytes.
The header value for a 256-byte transfer is
\verb+0xFC00+ = \verb+111_111_00_0000_0000b+, which indicates
that the write and read scan lengths are equal to the scan length,
and the scan length is 256-bytes.

The \verb+master_read/write_memory+ commands generate a JTAG
transaction containing as much data as possible (to maximize
the transfer rate). The header for the {\em write} transaction
uses the next greater scan length needed to fit the encoded
data (in increments of 256-bytes). The header always has
the 6 MSBs set, to indicate that the write and read data length
is set to the same length as the scan length.
The header for {\em read} transactions is slightly different, in
that the read command can be encoded in less than 256-bytes,
so the write scan length (the 3-MSBs) changes to 001b (256-bytes)
once the amount of read data exceeds 256-bytes, i.e., 
read transfers of up to 256-bytes use a header of \verb+0xFC00+, 
then the header changes to \verb+0x3C01+ = \verb+001_111_00_0000_0001b+,
which indicates a write (command) scan length of 256-bytes,
a read scan length equal to the scan length, and a scan 
length of 512-bytes. For longer read transactions, the header
changes to \verb+0x3C02+ for 768-bytes, \verb+0x3C03+ for 1024-bytes, 
etc. (where the number of bytes in the scan is the number of
encoded bytes, not the number of bytes passed via
the Tcl procedure argument).

Even though the write (command) scan length is less than the
read scan length, the host still has to send data down the JTAG
interface (since JTAG transactions are both a write and a read).
When the write scan counter (in the JTAG-to-Avalon-ST bridge)
terminates, the hardware ignores the data on TDI and inserts 
JTAG {\em IDLE} codes automatically. SignalTap II traces
show TDI going low when the write counter terminates,
indicating that the SystemConsole command is sending \verb+0x00+
data bytes. This logic is overly complicated and redundant;
the host has to send JTAG data bytes anyway, so it may as well send
{\em IDLE} codes, with the header representing just the
scan length (there is no need for the header to distinguish between the
read, write, and scan length). Actually, the requirement for the
JTAG-to-Avalon-ST communications to occur in blocks of 256-bytes
is unnecessary, but that is the subject of another document.

\end{itemize}
%
The SystemConsole \verb+master+ service 
\verb+master_read/write_32+ and \verb+master_read/write_memory+
procedures were re-implemented using the \verb+bytestream+
service procedures. Signal Tap II traces of various 
lengths confirmed the operation of the procedures.

The project simulation testbench, \verb+jtag_to_avalon_mm_tb.sv+,
uses the Verilog tasks in the JTAG SLD node to access the
JTAG-to-Avalon-ST bridge control and status registers,
and to generate various JTAG-to-Avalon-MM master transactions.
This testbench can be used as the basis for simulation testbenches
for testing Qsys and SOPC Systems containing the
JTAG-to-Avalon-MM component.

% -----------------------------------------------------------------
% JTAG-to-Avalon-MM master procedures
% -----------------------------------------------------------------
%
\begin{table}[t]
\caption{SystemConsole JTAG-to-Avalon-MM ({\tt master}) service procedures.}
\label{tab:systemconsole_master_procedures}
\begin{center}
\begin{tabular}{|l|p{43mm}|p{55mm}|}
\hline
Procedure & Arguments & Description\\
\hline\hline
&&\\
\verb+master_write_8+ & {\tt <handle> <addr>\newline <list of 8-bit values>} & 
Write the list of 8-bit values, starting from the specified address,
using 8-bit accesses.\\
&&\\
\verb+master_write_16+ & {\tt <handle> <addr>\newline <list of 16-bit values>} &
Write the list of 16-bit values, starting from the specified 
16-bit aligned address, using 16-bit accesses.\\
&&\\
\verb+master_write_32+ & {\tt <handle> <addr>\newline <list of 32-bit values>} &
Write the list of 32-bit values, starting from the specified 
32-bit aligned address, using 32-bit accesses.\\
&&\\
\verb+master_write_memory+ & {\tt <handle> <addr>\newline <list of 8-bit bytes>} &
Write the list of bytes, starting from the specified 
address.\\
&&\\
\hline
&&\\
\verb+master_read_8+ & {\tt <handle> <addr>\newline <number of 8-bit values to read>} & 
Read the requested number of 8-bit values, starting from the specified address,
using 8-bit accesses.\\
&&\\
\verb+master_read_16+ & {\tt <handle> <addr>\newline <number of 16-bit values to read>} &
Read the requested number of 16-bit values, starting from the specified 
16-bit aligned address, using 16-bit accesses.\\
&&\\
\verb+master_read_32+ & {\tt <handle> <addr>\newline <number of 32-bit values to read>} &
ead the requested number of 32-bit values, starting from the specified 
32-bit aligned address, using 32-bit accesses.\\
&&\\
\verb+master_read_memory+ & {\tt <handle> <addr>\newline <number of bytes to read>} &
ead the requested number of bytes, starting from the specified 
address.\\
&&\\
\hline
\end{tabular}
\end{center}
\end{table}
% -----------------------------------------------------------------

% -----------------------------------------------------------------
% SignalTap II logic analyzer traces
% -----------------------------------------------------------------
%
% The traces were created using;
% * Shift-Ctrl-PrintScreen
% * Crop using Corel PhotoPaint
% * Convert to bmp and back to png using png2bmp and bmp2png.
%   If you do not do this, you get an exception from PCTex.
% * Paste into Inkscape without scaling the bitmap.
% * Add overlay text.
%
% The notes in my vjtag.tex document, show how to crop using
% MATLAB.
%
\begin{landscape}
\begin{figure}
  \centering
  \includegraphics[width=210mm]
  {figures/jtag_to_avalon_mm_write_0x55667788_start.pdf}
  \caption{JTAG-to-Avalon-MM 32-bit master write (byte-stream start);
  (address, data) = ({\tt 0x11223344}, {\tt 0x55667788}).
  The Avalon-ST {\em command} byte-stream contains the Avalon-MM
  write command bytes in Table~\ref{tab:jtag_to_avalon_mm_write_packet},
  with the {\em write, incrementing} transaction code from
  Table~\ref{tab:jtag_to_avalon_mm_transaction_codes}, and the
  packet protocol codes from Table~\ref{tab:jtag_to_avalon_mm_protocol_codes}.}
  \label{fig:jtag_to_avalon_mm_write_start}
\end{figure}
\end{landscape}

\begin{landscape}
\begin{figure}
  \centering
  \includegraphics[width=210mm]
  {figures/jtag_to_avalon_mm_write_0x55667788_end.pdf}
  \caption{JTAG-to-Avalon-MM 32-bit master write (byte-stream end);
  (address, data) = ({\tt 0x11223344}, {\tt 0x55667788}).
  The Avalon-ST {\em response} byte-stream contains the Avalon-MM
  write response bytes in Table~\ref{tab:jtag_to_avalon_mm_write_packet},
  with the {\em write, incrementing} transaction code from
  Table~\ref{tab:jtag_to_avalon_mm_transaction_codes} (with the MSB set),
  and the packet protocol codes from 
  Table~\ref{tab:jtag_to_avalon_mm_protocol_codes}.}
  \label{fig:jtag_to_avalon_mm_write_end}
\end{figure}
\end{landscape}

\begin{landscape}
\begin{figure}
  \centering
  \includegraphics[width=210mm]
  {figures/jtag_to_avalon_mm_read_0x55667788_start.pdf}
  \caption{JTAG-to-Avalon-MM 32-bit master read (byte-stream start);
  (address, data) = ({\tt 0x11223344}, {\tt 0x55667788}).
  The Avalon-ST {\em command} byte-stream contains the Avalon-MM
  read command bytes in Table~\ref{tab:jtag_to_avalon_mm_read_packet},
  with the {\em read, incrementing} transaction code from
  Table~\ref{tab:jtag_to_avalon_mm_transaction_codes}, and the
  packet protocol codes from Table~\ref{tab:jtag_to_avalon_mm_protocol_codes}.}
  \label{fig:jtag_to_avalon_mm_read_start}
\end{figure}
\end{landscape}

\begin{landscape}
\begin{figure}
  \centering
  \includegraphics[width=210mm]
  {figures/jtag_to_avalon_mm_read_0x55667788_end.pdf}
  \caption{JTAG-to-Avalon-MM 32-bit master read (byte-stream end);
  (address, data) = ({\tt 0x11223344}, {\tt 0x55667788}).
  The Avalon-ST {\em response} byte-stream contains the Avalon-MM
  read response bytes in Table~\ref{tab:jtag_to_avalon_mm_read_packet},
  with the {\em read, incrementing} transaction code from
  Table~\ref{tab:jtag_to_avalon_mm_transaction_codes} (with the MSB set),
  and the packet protocol codes from 
  Table~\ref{tab:jtag_to_avalon_mm_protocol_codes}.}
  \label{fig:jtag_to_avalon_mm_read_end}
\end{figure}
\end{landscape}

\begin{landscape}
\begin{figure}
  \centering
  \includegraphics[width=210mm]
  {figures/jtag_to_avalon_mm_write_memory.pdf}
  \caption{JTAG-to-Avalon-MM master write memory (two 32-bit writes);
  (address, data) = ({\tt 0x11223344}, \{{\tt 0x55, 0x66, 0x77 0x88, 0x99, 0xAA, 0xBB, 0xCC}\}).
  The Avalon-ST {\em command} byte-stream contains the Avalon-MM
  write command bytes in Table~\ref{tab:jtag_to_avalon_mm_write_packet},
  with the {\em write, incrementing} transaction code from
  Table~\ref{tab:jtag_to_avalon_mm_transaction_codes}, and the
  packet protocol codes from Table~\ref{tab:jtag_to_avalon_mm_protocol_codes}.}
  \label{fig:jtag_to_avalon_mm_write_memory}
\end{figure}
\end{landscape}

\begin{landscape}
\begin{figure}
  \centering
  \includegraphics[width=210mm]
  {figures/jtag_to_avalon_mm_read_memory.pdf}
  \caption{JTAG-to-Avalon-MM 32-bit master read memory (two 32-bit reads);
  (address, data) = ({\tt 0x11223344}, \{{\tt 0x55, 0x66, 0x77 0x88, 0x55, 0x66, 0x77 0x88}\}).
  The Avalon-ST {\em command} byte-stream contains the Avalon-MM
  read command bytes in Table~\ref{tab:jtag_to_avalon_mm_read_packet},
  with the {\em read, incrementing} transaction code from
  Table~\ref{tab:jtag_to_avalon_mm_transaction_codes}, and the
  packet protocol codes from Table~\ref{tab:jtag_to_avalon_mm_protocol_codes}.}
  \label{fig:jtag_to_avalon_mm_read_memory}
\end{figure}
\end{landscape}

\begin{landscape}
\begin{figure}
  \centering
  \includegraphics[width=210mm]
  {figures/jtag_to_avalon_mm_write_0x7D7C7B7A.pdf}
  \caption{JTAG-to-Avalon-MM 32-bit master write (byte-stream start);
  (address, data) = ({\tt 0x11223344}, {\tt 0x7D7C7B7A}).
  The Avalon-MM write data contains all of the packet protocol codes in
  Table~\ref{tab:jtag_to_avalon_mm_protocol_codes} {\em as data}.
  The logic analyzer trace shows the {\em ESCAPE} code and the
  XOR modified data in both the JTAG and Avalon-ST byte-streams.}
  \label{fig:jtag_to_avalon_mm_write_0x7D7C7B7A}
\end{figure}
\end{landscape}

\begin{landscape}
\begin{figure}
  \centering
  \includegraphics[width=210mm]
  {figures/jtag_to_avalon_mm_write_0x4D4A.pdf}
  \caption{JTAG-to-Avalon-MM 32-bit master write (byte-stream start);
  (address, data) = ({\tt 0x11223344}, {\tt 0x4D4A}).
  The Avalon-MM write data contains all of the Avalon-ST protocol codes in
  Table~\ref{tab:jtag_to_avalon_st_protocol_codes} {\em as data}.
  The logic analyzer trace shows the {\em ESCAPE} code and the
  XOR modified data in the JTAG byte-stream.}
  \label{fig:jtag_to_avalon_mm_write_0x4D4A}
\end{figure}
\end{landscape}

\clearpage
\appendix
% =================================================================
\section{Project synthesis and simulation}
% =================================================================
\label{app:build}

% -----------------------------------------------------------------
% Tool versions
% -----------------------------------------------------------------
%
\begin{table}[t]
\caption{JTAG-to-Avalon-MM ({\tt jtag\_to\_avalon\_mm}) project source.}
\label{tab:tool_versions}
\begin{center}
\begin{tabular}{|c|c|}
\hline
Quartus II & Modelsim\\
\hline\hline
&\\
10.1 and 10.1sp1 (Full) & 6.6c (ASE)\\
11.0 and 11.0sp1 (Full) & 6.6d (ASE)\\
11.1 and 11.1sp1 (Full) & 10.0c (ASE)\\
&\\
\hline
\end{tabular}
\end{center}
\end{table}
% -----------------------------------------------------------------

This appendix describes how to synthesize and simulate each of
the projects described in this document.
Table~\ref{tab:tool_versions} shows the tool versions each of the
scripts has been tested with. The tools were tested under
Windows XP, Windows 7, and Linux (Ubuntu 11.10 and Centos 6.2).

% -----------------------------------------------------------------
\subsection{Synthesis}
% -----------------------------------------------------------------

\begin{enumerate}
\item Start Quartus II.
\item If the Tcl console is not visible, add it to the GUI via
{\em View}$\rightarrow${\em Utility Windows}$\rightarrow${\em Tcl Console}.
\item Click on the Tcl Console command line.
\item Change directory to the top-level of the project source, eg.,
%
\begin{verbatim}
cd {C:\altera_jtag_to_avalon_analysis\hdl\jtag_node}
\end{verbatim}
where the path was copied from Windows Explorer into the console.
The path is converted to a Tcl list using the parentheses \verb+{+ 
and \verb+}+, ensuring the windows path separators are handled correctly.
%
\item Run the synthesis script
%
\begin{verbatim}
source scripts/synth.tcl
\end{verbatim}
%
Table~\ref{tab:project_synthesis_results} shows the synthesis results
for the three projects discussed in this document.
%
\item (Optional) Enable the SignalTap II trace; 
{\em Assignments}$\rightarrow${\em Settings}, select
{\em SignalTap II Logic Analyzer}, and check the checkbox
(the \verb+bemicro_sdk.stp+ file is already setup).
%
\item Re-synthesize the design by clicking the GUI `play' button.
%
\item Download the design to the BeMicro-SDK; LED8 will start to flash.
%
\item Start SystemConsole; {\em Tools}$\rightarrow${\em Transceiver Toolkit}
%
\item Load the custom Tcl procedures created for the project
%
\begin{verbatim}
source ../scripts/jtag_cmds_sc.tcl
\end{verbatim}
%
(the SystemConsole shell starts in the Quartus work directory, which
is located in the project directory, so a relative path can be used to
access the scripts directory).
%
\item Start the SignalTap II logic analyzer.
%
\item Use the procedures in \verb+jtag_cmds_sc.tcl+ to generate transactions
and trace them using SignalTap II.
%
\end{enumerate}

% -----------------------------------------------------------------
% Synthesis results
% -----------------------------------------------------------------
%
\begin{table}[t]
\caption{JTAG project synthesis results (for Quartus 11.0sp1 full-edition).}
\label{tab:project_synthesis_results}
\begin{center}
\begin{tabular}{|l||c|c|c||c|}
\hline
        & \multicolumn{4}{c|}{Resource Usage}\\
\cline{2-5}
Project & JTAG Interface & SLD Hub & User Logic & Total\\
\hline\hline
&&&&\\
\verb+jtag_node+         &   1 LC  & 99 LCs &  49 LCs & 149 LCs\\
&&&&\\
\verb+jtag_to_avalon_st+ & 426 LCs & 99 LCs &  51 LCs & 576 LCs\\
&&&&\\
\verb+jtag_to_avalon_mm+ & 848 LCs & 99 LCs & 113 LCs & 1060 LCs\\
                         & + 512-bits RAM &&& + 512-bits RAM\\
&&&&\\
\hline
\end{tabular}
\end{center}
\end{table}
% -----------------------------------------------------------------

% -----------------------------------------------------------------
\subsection{Simulation}
% -----------------------------------------------------------------

\begin{enumerate}
\item Start Modelsim.
\item Change directory to the top-level of the project source, eg.,
%
\begin{verbatim}
cd {C:\altera_jtag_to_avalon_analysis\hdl\jtag_node}
\end{verbatim}
where the path was copied from Windows Explorer into the console.
The path is converted to a Tcl list using the parentheses \verb+{+ 
and \verb+}+, ensuring the windows path separators are handled correctly.
%
\item Run the simulation script
%
\begin{verbatim}
source scripts/sim.tcl
\end{verbatim}
%
The simulation script creates a Tcl procedure with the same
name as the testbench, eg., \verb+jtag_node_tb+. 

\item Run the testbench procedure.
The procedure will add signals to the wave window and the Tcl
console will output messages from the simulation.
%
\end{enumerate}

\clearpage
% =================================================================
\section{JTAG-to-Avalon source, software, and hardware bugs}
% =================================================================
\label{app:bugs}

The following are a list of bugs in the Quartus II JTAG-to-Avalon
software and source code;
%
\begin{enumerate}
\item In the JTAG-to-Avalon-MM source file

\verb+altera_jtag_avalon_master/altera_jtag_avalon_master_pli_off.v+

the parameter \verb+DOWNSTREAM_FIFO_SIZE+ is passed as 6, whereas it
should be passed as the FIFO size (64-bytes). The
JTAG-to-Avalon-ST source file

\verb+altera_avalon_jtag_phy/altera_jtag_streaming.v+

calculates the parameter \verb+DOWNSTREAM_ENCODED_SIZE+
as the $\floor(\log_2(\text{\tt DOWNSTREAM\_FIFO\_SIZE}))$, and
then makes that value available via the {\em INFO} register.
Hardware tests show the downstream FIFO register bits are
read-back as $\floor(\log_2(6)) = 2$, rather than the
expected 6.
%
\item The Verilog simulation tasks in the source file

\verb+altera_avalon_jtag_phy/altera_jtag_sld_node.v+

Use a continuously running JTAG clock, TCK. This does not
reflect the hardware implementation, and hides a subtle bug with
respect to the use of the Virtual JTAG one-hot state signals
\verb+virtual_state_e1dr+ and \verb+virtual_state_udr+.

\item The JTAG control registers in

\verb+altera_avalon_jtag_phy/altera_jtag_streaming.v+

Incorrectly use the Virtual JTAG one-hot state \verb+virtual_state_udr+
to update the registers. The logic {\em should} be using the
\verb+virtual_state_e1dr+ one-hot state as the enable control.
The \verb+virtual_state_e1dr+ signal pulses for a single clock
after the JTAG state machine exits the shift-DR state (\verb+virtual_state_sdr+).
The update-DR state (\verb+virtual_state_udr+) is used to indicate
when data register have been updated, it should not be used to
update the registers. This causes an issue with the \verb+resetrequest+
control, in that you can write to the register and it has no effect
until the next JTAG transaction. This occurs due to the fact that
the JTAG TAP ends in the \verb+virtual_state_udr+ state, and there
are no TCK edges to {\em load} the registers until the {\em next}
JTAG transaction.

This error is not obvious in the simulation due to the fact that
a free-running TCK is used. This generates a clock in the 
\verb+virtual_state_udr+ state  causing the resetrequest output
to update. This sequence is not reflected in the hardware, where
TCK is stopped until the next JTAG transaction.

\item The SystemControl \verb+bytestream_send+ command only ever
sends 1022-bytes (\verb+0x3FE+ bytes) per 1024-bytes (\verb+0x400+ bytes)
transaction (where the transaction length is determined by the
16-bit byte-stream header). Perhaps the authors of the SystemConsole
code incorrectly assumed that the 2-byte header was included in
the length?

\item The JTAG-to-Avalon-MM SystemConsole \verb+master+ service,
\verb+master_write/read_8/16/32+ procedures generate
256-byte JTAG-to-Avalon-ST transactions for each Avalon-MM
transaction, eg., a call to \verb+master_write_32+ with a list
of 32-bit data values generates multiple JTAG transactions to
perform the 32-bit Avalon-MM bus writes.
The overhead of encoding multiple 32-bit transaction results in
a significant loss in performance over the JTAG interface.

The JTAG-to-Avalon-MM SystemConsole \verb+master+ service,
\verb+master_write/read_memory+ procedures can be used to generate
multiple 32-bit Avalon-MM bus transactions from a single
JTAG-to-Avalon-ST transaction. 

The Quartus II handbook should document the performance
difference between the SystemConsole \verb+master+ procedures.

\item The SignalTap II traces in Figures~\ref{fig:jtag_node_vir}
and~\ref{fig:jtag_node_vdr} show glitches on the JTAG interface.
This implicates the USB-Blaster as the source.

\item The SignalTap II trace in Figure~\ref{fig:jtag_node_vdr}
show an invalid Virtual JTAG one-hot state assertion
(\verb+virtual_state_cir+ during a Virtual shift-DR sequence).
This error must be in the implementation of the Virtual JTAG
logic (the source of which is not provided by Altera).

\item In versions of Quartus II prior to version 11.1sp1,
SystemConsole does not support the Tcl \verb+fileevent+ 
procedure. This procedure is {\em required} to implement a
Tcl server that can support multiple clients. This type
of server is {\em necessary} to provide JTAG-to-Avalon-MM
access to software that is not written using SystemConsole.

\end{enumerate}


\clearpage
%------------------------------------------------------------------
% Do the bibliography
%------------------------------------------------------------------
%Note, you can't have spaces in the list of bibliography files
%
\bibliography{refs}
\bibliographystyle{plain}

%------------------------------------------------------------------
\end{document}











